%!TEX TS-program = xelatex
%!TEX encoding = UTF-8 Unicode
% Awesome CV LaTeX Template for Cover Letter
%
% This template has been downloaded from:
% https://github.com/posquit0/Awesome-CV
%
% Authors:
% Claud D. Park <posquit0.bj@gmail.com>
% Lars Richter <mail@ayeks.de>
%
% Template license:
% CC BY-SA 4.0 (https://creativecommons.org/licenses/by-sa/4.0/)
%


%-------------------------------------------------------------------------------
% CONFIGURATIONS
%-------------------------------------------------------------------------------
% A4 paper size by default, use 'letterpaper' for US letter
\documentclass[12pt, a4paper]{awesome-cv}

% Configure page margins with geometry
\geometry{left=1.4cm, top=.8cm, right=1.4cm, bottom=1.8cm, footskip=.5cm}
%\setleftheader{\includegraphics[width=5cm]{pics/bewerbungsfoto.jpg}}
% Specify the location of the included fonts
\fontdir[fonts/]

% Color for highlights
% Awesome Colors: awesome-emerald, awesome-skyblue, awesome-red, awesome-pink, awesome-orange
%                 awesome-nephritis, awesome-concrete, awesome-darknight
\colorlet{awesome}{awesome-basti}
% Uncomment if you would like to specify your own color
% \definecolor{awesome}{HTML}{CA63A8}

% Colors for text
% Uncomment if you would like to specify your own color
% \definecolor{darktext}{HTML}{414141}
% \definecolor{text}{HTML}{333333}
% \definecolor{graytext}{HTML}{5D5D5D}
% \definecolor{lighttext}{HTML}{999999}

% If you would like to change the social information separator from a pipe (|) to something else
\renewcommand{\acvHeaderSocialSep}{\quad\textbar\quad}


%-------------------------------------------------------------------------------
%	PERSONAL INFORMATION
%	Comment any of the lines below if they are not required
%-------------------------------------------------------------------------------
\name{Sebastian}{Nagel}
\position{B.Sc.Elektro- und Informationstechnik}
\address{Geibelstr. 10, 81679 Munich, Germany}

\mobile{(+49) 157/37321816} 
\email{basti.nagel@tum.de}
\github{GitHub}
\linkedin{LinkedIn}

%-------------------------------------------------------------------------------
%	LETTER INFORMATION
%	All of the below lines must be filled out
%-------------------------------------------------------------------------------
% The company being applied to
\recipient
  {Company Recruitment Team}
  {Robominds}
% The date on the letter, default is the date of compilation
\letterdate{\today}
% The title of the letter
\lettertitle{Bewerbung als Werkstudent Software Development}
% How the letter is opened

\letteropening{\textbf{Sehr geehrter Herr Däubler,}}
\letterclosing{Mit besten Grüßen, }
\letterenclosure[Beigef\"ugt]{Curriculum Vitae}


%-------------------------------------------------------------------------------
\begin{document}
\makecvheader
\makecvfooter
  {\today}
  {Sebastian Nagel~~~·~~~Bewerbungsschreiben}
  {}

\medskip
\makelettertitle

%-------------------------------------------------------------------------------
%	LETTER CONTENT
%-------------------------------------------------------------------------------
%
%
%-------------------------------------------------------------------------------
\begin{cvletter}

Bei einem Besuch der Internetplatform ``stackoverflow" bin ich zufällig auf ein Job Angebot Ihrer Firma gestoßen. Im kommenden Sommersemester fange ich mein Master Studium ``Robotics, Cognition and Intelligence" an der TU München an. Da ich aktuell noch auf der Suche nach einem Nebenjob bin, der optimalerweise in direktem Zusammenhang mit meinem Studium steht, habe ich mich dazu entschieden mich bei Ihnen als Werkstudent zu bewerben.

Mein Bachelorstudium habe ich im Bereich der Elektrotechnik an der TU München im April vergangenen Jahres abgeschlossen.
\\Während meines Studiums habe ich gemerkt, dass mich Programmieren und die Arbeit mit dem Computer am meisten begeistert. Meinen ersten Kontakt mit Software Entwicklung hatte ich während meiner Beschäftigung als wissenschaftliche Hilfskraft an dem Lehrstuhl für Neurowissenschaftliche Systemtheorie, parallel zu meinem Studium ab dem 3. Semester. Eine meiner Arbeiten umfasste dabei die Programmierung von Robotern und Microcontrollern, teilweise unter der Verwendung von Linux.
\\Das Interesse, hardwarenah zu programmieren, hat sich dann auch bei der Themenauswahl meiner Bachelorarbeit ausgewirkt, bei der ich ein kryptographisches Verschlüsselungsverfahren auf einem 16-bit Mikrocontroller implementiert habe. Die Arbeit war Teil einer Vorentwicklung für BMW DriveNow und wurde bei der Firma Remes in Erding angefertigt.
\\Nach Beendigung meiner Arbeit und meines Abschlusses habe ich an dem Projekt noch für ein paar Monate in Vollzeit als “Systementwickler” weitergearbeitet, sodass ich auch schon ein wenig Berufserfahrung sammeln konnte. Während dieser Zeit bin ich zu der Entscheidung gekommen, dass mir der Bachelor Abschluss nicht ausreicht und ich mich noch weiterbilden will. Deshalb habe ich mich dann für das oben erwähnte Master Studium eingeschrieben. 
\\Als “Tech-Skill” führen sie auf Ihrer website C++. Während meines Studiums habe ich zwei C++ Kurse belegt, sodass ich Erfahrungen im Bereich der objektorientierten Programmierung aufweisen kann. Bei einem der beiden Kurse habe ich vergangenes Jahr ein Angeobt des Lehrstuhls angenommen, als Tutor für neue Teilnehmer weiterzuarbeiten. Darüber hinaus habe ich auch ein freizeitliches Interesse daran meine Kentnisse in C++ und der Objekt-Orientiereten-Programmierung auszbauen.
\\Mit Unit Tests, Python oder PHP habe ich bis jetzt allerdings leider noch keine bis kaum Erfahrungen gemacht. Vor allem aber mit Unit Tests möchte ich mich schon seit längerem auseinandersetzen. 
\\Ich bin ein motivierter, lernbegeisterter, aber noch recht unerfahrener, Programmierer. Für mich wäre der Nebenjob eine optimale Möglichkeit neue, interessante Dinge zu lernen und diese auch dirket in einem praktischen Umfeld anzuwenden.
\\Arbeitsbeginn wäre für mich ab ab sofort möglich. Für Rückfragen oder zusätzliche Infos stehe ich Ihnen gerne unter den oben angegebenen Kontaktmöglichkeiten bzw. in einem persönlichen Gespräch zur Verfügung.
\end{cvletter}


%-------------------------------------------------------------------------------
% Print the signature and enclosures with above letter informations
\makeletterclosing

\end{document}
